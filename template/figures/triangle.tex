% \newcommand*\calcBinom[2] {\xinttheexpr subs(subs(x!//y!//(x-y)!,y=#2), x=#1)\relax }
\begin{tikzpicture}[yscale=1/2]
	\tikzmath{
		% params
		\tf = 0.7;
		\steps = 5;
		% computed
		\t = \tf * \steps + 1;
		\lines = ceil(\steps/\tf);
		\lineMax = \lines-1;
		% spacing
		\aBit = 0.1;
	};
	% axes
	% x
	\draw[-] (-0.2,0) -- (\steps,0) node[right] {$t$};
	% y
	\draw[-] (0,-0.4) node[left] {0} -- (0,\lines) node[left] {$k$};
	% braces
	% tf
	\draw[-, decorate, decoration = {brace}, yshift=0.1cm, color=red] (0, 1) -- (\tf, 1) node[midway,yshift=0.4cm] {$T_F$};
	% tn
	\draw[-, decorate, decoration = {brace}, yshift=-0.1cm, color=blue] (1, 0) -- (0, 0) node[midway,yshift=-0.4cm] {$T_N$};
	% bricks
	\foreach \lineIndex in {0, ..., \lineMax} {
			\tikzmath{
				int \fullStepsPerLine;
				\stepsPerLine = \steps-\lineIndex*\tf;
				\fullStepsPerLine = floor(\stepsPerLine);
				\fullStepsPerLineMax = \fullStepsPerLine-1;
			}
			% full bricks
			\ifnum \fullStepsPerLine>0
				\foreach \stepIndex in {0,...,\fullStepsPerLineMax} {
						\tikzmath{
							\start = \stepIndex + \lineIndex * \tf;
						}
						\draw (\start, \lineIndex) rectangle (\start + 1, \lineIndex + 1);
					}
			\fi
			% partial bricks
			\tikzmath{
				\start = \lineIndex*\tf+\fullStepsPerLine;
			}
			\draw[-] (\steps,\lineIndex) -- (\start,\lineIndex) --
			(\start, \lineIndex + 1) -- (\steps, \lineIndex + 1);
			% plots
			\draw[-, domain=0:\stepsPerLine, samples=100, thin, gray] plot (\lineIndex * \tf+\x,{sin(\x*2*pi r)/3+\lineIndex+1/2});
		};
\end{tikzpicture}