\newcommand{\tri}[1]{
	\begin{tikzpicture}[yscale=1/2]
		\tikzmath{
			% params
			\tf = 0.7;
			\steps = 5;
			% computed
			\t = \tf * \steps + 1;
			\lines = ceil(\steps/\tf);
			\lineMax = \lines-1;
		};
		% axes
		% x
		\draw[-] (-0.2,0) -- (\steps,0) node[right] {$t$};
		% y
		\draw[-] (0,-0.4) node[left] {0} -- (0,\lines) node[left] {$k$};
		% braces
		% tf
		\draw[-, decorate, decoration = {brace}, yshift=0.1cm, color=red] (0, 1) -- (\tf, 1) node[midway,yshift=0.4cm] {$T_F$};
		% tn
		\draw[-, decorate, decoration = {brace}, yshift=-0.1cm, color=blue] (1, 0) -- (0, 0) node[midway,yshift=-0.4cm] {$T_N$};
		% permutations arrows
		\ifthenelse{\equal{#1}{permutations}}{
			% \begin{scope}[yshift=0.5cm, xshift=0.5cm]
			% 	\draw[->, red] (0,0) -- (1, 0) -- (\tf+1, 1);
			% 	\draw[->, green] (0,0) -- (\tf, 1) -- (\tf+1, 1);
			% \end{scope}
			\begin{scope}[yshift=0.5cm, xshift=0.5cm]
				\draw[->, green, yshift=-0.1cm] (0,0) -- (2, 0) -- (\tf+2, 1);
				\draw[->, blue] (0,0) -- (1, 0) -- (\tf+1, 1) -- (\tf+2, 1);
				\draw[->, red, yshift=0.1cm] (0,0) -- (\tf, 1) -- (\tf+2, 1);
			\end{scope}
		}{}
		% bricks
		\foreach \lineIndex in {0, ..., \lineMax} {
				\tikzmath{
					int \fullStepsPerLine;
					\stepsPerLine = \steps-\lineIndex*\tf;
					\fullStepsPerLine = floor(\stepsPerLine);
					\fullStepsPerLineMax = \fullStepsPerLine-1;
				}
				% full bricks
				\ifnum \fullStepsPerLine>0
					\foreach \stepIndex in {0,...,\fullStepsPerLineMax} {
							\tikzmath{
								\start = \stepIndex + \lineIndex * \tf;
							}
							% permutation numbers
							\ifthenelse{\equal{#1}{permutations}}{
								\draw (\start, \lineIndex) rectangle node[] {\calcBinom{\stepIndex+\lineIndex}{\stepIndex}} (\start + 1, \lineIndex + 1);
							}{
								\ifthenelse{\equal{#1}{losses}}{
									\tikzmath{
										int \lossIndex;
										\lossIndex = \lineIndex+\stepIndex;
									}
									\draw (\start, \lineIndex) rectangle node[] {$R^{\lossIndex}$} (\start + 1, \lineIndex + 1);
								}{
									\draw (\start, \lineIndex) rectangle (\start + 1, \lineIndex + 1);
								}
							}
						}
				\fi
				% partial bricks
				\tikzmath{
					\start = \lineIndex*\tf+\fullStepsPerLine;
				}
				\draw[-] (\steps,\lineIndex) -- (\start,\lineIndex) --
				(\start, \lineIndex + 1) -- (\steps, \lineIndex + 1);
				% plots
				\ifthenelse{\equal{#1}{plot}}{
					\draw[-, domain=0:\stepsPerLine, samples=100, thin, gray] plot (\lineIndex * \tf+\x,{sin(\x*2*pi r)/3+\lineIndex+1/2});
				}
			}
	\end{tikzpicture}
}