\documentclass{sigchi}

% Use this section to set the ACM copyright statement (e.g. for
% preprints).  Consult the conference website for the camera-ready
% copyright statement.

% Copyright
\CopyrightYear{2024}
%\setcopyright{acmcopyright}
\setcopyright{acmlicensed}
%\setcopyright{rightsretained}
%\setcopyright{usgov}
%\setcopyright{usgovmixed}
%\setcopyright{cagov}
%\setcopyright{cagovmixed}
% DOI
\doi{https://doi.org/10.1145/3313831.XXXXXXX}
% ISBN
\isbn{978-1-4503-6708-0/20/04}
%Conference
\conferenceinfo{CHI'20,}{April  25--30, 2020, Honolulu, HI, USA}
%Price
\acmPrice{\$15.00}

% Use this command to override the default ACM copyright statement
% (e.g. for preprints).  Consult the conference website for the
% camera-ready copyright statement.

%% HOW TO OVERRIDE THE DEFAULT COPYRIGHT STRIP --
%% Please note you need to make sure the copy for your specific
%% license is used here!
\toappear{
	% Permission to make digital or hard copies of all or part of this work
	% for personal or classroom use is granted without fee provided that
	% copies are not made or distributed for profit or commercial advantage
	% and that copies bear this notice and the full citation on the first
	% page. Copyrights for components of this work owned by others than the author
	% must be honored. Abstracting with credit is permitted. To copy
	% otherwise, or republish, to post on servers or to redistribute to
	% lists, requires prior specific permission and/or a fee.% Request
	% permissions from \href{mailto:ellinas@kth.se}{ellinas@kth.se}. \\
	% \emph{CHI '16},  May 07--12, 2016, San Jose, CA, USA \\
	% ACM xxx-x-xxxx-xxxx-x/xx/xx\ldots \$15.00 \\
	% DOI: \url{http://dx.doi.org/xx.xxxx/xxxxxxx.xxxxxxx}
	}
	% \toappear{}

% Arabic page numbers for submission.  Remove this line to eliminate
% page numbers for the camera ready copy
% \pagenumbering{arabic}

% Load basic packages
\usepackage{balance}       % to better equalize the last page
\usepackage{graphics}      % for EPS, load graphicx instead 
\usepackage[T1]{fontenc}   % for umlauts and other diaeresis
\usepackage{txfonts}
\usepackage{mathptmx}
\usepackage[pdflang={en-US},pdftex]{hyperref}
\usepackage{color}
\usepackage{booktabs}
\usepackage{textcomp}
\usepackage{tikz}
\usepackage{amsmath}
\usepackage{xintexpr}
\usepackage{url}
\usepackage{siunitx}
\usetikzlibrary {arrows, shapes, automata, positioning, math, calc, external, decorations.pathreplacing, calligraphy}

\usepackage{ifthen}
\newcommand*\calcBinom[2] {\xinttheexpr subs(subs(x!//y!//(x-y)!,y=#2), x=#1)\relax } 
\begin{tikzpicture}
	\tikzmath{
		\p = 0.3;
		\invP = 1-\p;
		\mMax = 5;
		\mPos = \mMax + 1;
		\n = \invP * \mMax + 1;
		\shift = -(\n - (\mPos + 1))/2;
	};
	\foreach \m  in {0, ..., \mPos} {
		\tikzmath{
			\kSpace = \m*\invP-0.5;
		}
		\node[text width=2cm] at (0, \m/2+0.25) {$k=\m$};
		\foreach \k in {0,...,\m}{
			\tikzmath{
				\x = \m - \k * \p;
				% \bi = \calcBinom{\m}{\k};
				integer \mk;
				\mk = \m - \k;
			};

			% \draw (\x, \k / 2) rectangle node {$\m$} (\x + 1, \k / 2 + 0.5);
			\draw (\x, \k / 2) rectangle node {$\mk$ . $\k$} (\x + 1, \k / 2 + 0.5);
		};
	};
	\draw[fill=white, color=white] (\n, -0.1) rectangle (\mPos+1,\mPos/2+0.6);
	\path[thick, -, color=black] (\n, 0) -- (\n, \mPos/2+0.5);
	\draw[-, decorate, decoration = {brace}] (0,0.6) -- (\invP,0.6) node[midway,yshift=0.4cm] {$T_F$};
	\draw[-, decorate, decoration = {brace}] (1,-0.1) -- (0,-0.1) node[midway,yshift=-0.4cm] {$T_N$};
	\draw[-] (-1,0) -- (\n+1, 0);
	\node[branch, label={[xshift=-0.1cm, yshift=-0.7cm]$0$}] at (0, 0) {};
	\coordinate[label=below:$n$](time) at (\n, 0);
\end{tikzpicture}
\newcommand{\mmm}{\left\lfloor \frac{t - kT_F}{T_N} \right\rfloor}
\newcommand*\calcBinom[2] {\xinttheexpr subs(subs(x!//y!//(x-y)!,y=#2), x=#1)\relax }

\tikzset {
    sum/.style = {
        draw,
        shape = circle,
        font={$+$}
    },
		block/.style = {
			draw,
			shape = rectangle,
			minimum height = 2em,
			fill = white,
			align = center
		},
    multiply/.style args = {#1}{
        draw,
        regular polygon,
        regular polygon sides = 3,
        inner sep=0pt,
				fill = white,
        minimum size=25pt,
    },
    multiply right/.style args = {#1}{
        multiply={#1},
        shape border rotate=-90,
    },
    multiply left/.style args = {#1}{
        multiply={#1},
        shape border rotate=90,
    },
		multiply down/.style args = {#1}{
			multiply={#1},
			shape border rotate=180,
		},
		multiply up/.style args = {#1}{
			multiply={#1},
		},
    branch/.style = {
        draw,
        shape = circle,
        fill,
        inner sep=0pt,
        minimum size=4pt
    },
    every path/.style = {
        ->,
        >=stealth',
        line width=1pt,
        draw,
    }
}
% -{Latex[length=5mm]}
\pgfdeclarelayer{bg}    % declare background layer
\pgfsetlayers{bg,main}  % set the order of the layers (main is the standard layer)

\tikzset{
	beginning/.pic={
		\draw [pic actions] (0,0) -- (1,1) -- (3,1) -- (3,0) -- (0,0);
	}
}

\tikzset{
	end/.pic={
		\draw [pic actions] (0,0) -- (1,0) -- (0,1) -- (0,0);
	}
}

% Some optional stuff you might like/need.
\usepackage{microtype}        % Improved Tracking and Kerning
% \usepackage[all]{hypcap}    % Fixes bug in hyperref caption linking
\usepackage{ccicons}          % Cite your images correctly!
% \usepackage[utf8]{inputenc} % for a UTF8 editor only

\usepackage{todonotes}

% Paper metadata (use plain text, for PDF inclusion and later
% re-using, if desired).  Use \emtpyauthor when submitting for review
% so you remain anonymous.
\def\plaintitle{Performing string multiphonics on any pitched sound}
\def\plainauthor{Dimitrios Aatos Ellinas}
\def\emptyauthor{}
\def\plainkeywords{physical modelling; audio features}
\def\plaingeneralterms{Documentation, Standardization}

% llt: Define a global style for URLs, rather that the default one
\makeatletter
\def\url@leostyle{%
  \@ifundefined{selectfont}{
    \def\UrlFont{\sf}
  }{
    \def\UrlFont{\small\bf\ttfamily}
  }}
\makeatother
\urlstyle{leo}

% To make various LaTeX processors do the right thing with page size.
\def\pprw{8.5in}
\def\pprh{11in}
\special{papersize=\pprw,\pprh}
\setlength{\paperwidth}{\pprw}
\setlength{\paperheight}{\pprh}
\setlength{\pdfpagewidth}{\pprw}
\setlength{\pdfpageheight}{\pprh}

% Make sure hyperref comes last of your loaded packages, to give it a
% fighting chance of not being over-written, since its job is to
% redefine many LaTeX commands.
\definecolor{linkColor}{RGB}{6,125,233}
\hypersetup{%
  pdftitle={\plaintitle},
% Use \plainauthor for final version.
%  pdfauthor={\plainauthor},
  pdfauthor={\emptyauthor},
  pdfkeywords={\plainkeywords},
  pdfdisplaydoctitle=true, % For Accessibility
  bookmarksnumbered,
  pdfstartview={FitH},
  colorlinks,
  citecolor=black,
  filecolor=black,
  linkcolor=black,
  urlcolor=linkColor,
  breaklinks=true,
  hypertexnames=false
}

% create a shortcut to typeset table headings
% \newcommand\tabhead[1]{\small\textbf{#1}}

% End of preamble. Here it comes the document.
\begin{document}

\title{\plaintitle}

\numberofauthors{1}
\author{%
	\alignauthor{Dimitrios Aatos Ellinas\\
		\affaddr{Interactive Media Technology}\\
		\affaddr{KTH Royal Insitute of Technology}\\
		\affaddr{Stockholm, Sweden}\\
		\email{ellinas@kth.se}}\\
}

\maketitle

% \numberwithin{figure}{section}

\begin{abstract}
	Multiphonics and flageolet tones on string instruments are performed by lightly touching the vibrating string.
	This paper presents a physical model isolating the effect of the finger.
	The model can be applied as an audio effect to sounds of open, untouched strings, but also to arbitrary pitched sounds.
\end{abstract}


% ACM Classification

% \begin{CCSXML}
% <ccs2012>
% <concept>
% <concept_id>10003120.10003121</concept_id>
% <concept_desc>Human-centered computing~Human computer interaction (HCI)</concept_desc>
% <concept_significance>500</concept_significance>
% </concept>
% <concept>
% <concept_id>10003120.10003121.10003125.10011752</concept_id>
% <concept_desc>Human-centered computing~Haptic devices</concept_desc>
% <concept_significance>300</concept_significance>
% </concept>
% <concept>
% <concept_id>10003120.10003121.10003122.10003334</concept_id>
% <concept_desc>Human-centered computing~User studies</concept_desc>
% <concept_significance>100</concept_significance>
% </concept>
% </ccs2012>
% \end{CCSXML}

% \ccsdesc[500]{Human-centered computing~Human computer interaction (HCI)}
% \ccsdesc[300]{Human-centered computing~Haptic devices}
% \ccsdesc[100]{Human-centered computing~User studies}

% Author Keywords
\keywords{\plainkeywords}

% Print the classification codes
\printccsdesc

\section{Introduction}
% https://www.mdpi.com/2076-3417/6/5/143

% Consider a musician... while pitch and loudness... timbre is not.. a sound designer might be able to simulate... but simulations are limited... physical models preferable
% physical models are good at 
% physical models solve a problem different from the one we're concerned

Consider a musician playing a piece of music on an acoustic instrument and whishing to play the same piece on a synthesizer.
While pitch and loudness are readily controllable in both domains, it's not equally trivial to do equivalent timbral effects on acoustic and synthesized sounds.
A skilled sound designer might be able to program a synthesizer to simulate timbral techniques of an acoustic instrument, but will inevitably reach the limits of simulation.
The usual alternative to simulation is physical modelling synthesis, however, using previous research entails a challenge, as the task in question is not synthesis, but isolating an effect or technique and applying it to another arbitrary sound.
This paper tackles this challenge for the case of the lightly touched string, proposing an explicit formula.
% In order to tackle this challenge, this paper proposes an analysis-resynthesis approach, where a physical model is decomposed to a certain degree.
% In particular, isolating a timbral effect of an acoustic instrument and applying it to a synth.
% , e.g. using \textit{distortion} to emulate \textit{string popping} on a guitar.
% such as being unable to capture shuttle nuances and failing on more complex scenarios.

Lightly touching a vibrating string is a practice associated with a number of string instrument playing techniques.
Depending on the touching position and the touching agent, a different technique is performed.
Flageolet tones (harmonics) are performed by lightly touching a string with a finger on specific fractions of its length.
The produced sound has a pitch much higher than the original string and a different timbre, often compared to a flute or a whistle.
Multiphonics is a set of techniques that make single pitch sound sources produce sounds with more than one perceivable pitch.
On strings, multiphonics are performed by lightly touching the string at positions other than those producing harmonics.
Prepared strings is another technique where instead of a finger, a variety of objects such as rubbers, bolts or crocodile clips may be attached to the string, the position of the object might be either that of an multiphonic or a harmonic.
% (e.g. single strings and wind instruments)

Applying acoustic effects to synths or arbitrary sounds can be attractive for creative purposes, acoustic techniques can bring physical qualities to synthesized sounds.
Incorporating physicality in sound synthesis serves more than just aesthetics, a well modelled acoustic technique may involve recognizable parameters that can be utilized for information encoding, sonification and interaction design.

\section{Background}
% An example of this can be seen in robot voices, where models of the human vocal tract are combined with synthesized sounds totally unrelated to vocal folds.
Integrating physical qualities into synthesized sounds has been investigated in various ways, not only by using physical models.
Formant filters or models of the human vocal tract are typically applied to arbitrary synthesizers to produce robot voices.
The \textit{talking bass} (or \textit{YaYa bass}) is a synth used in dubstep, that combines a low pass filtering and sample rate reduction to create vocal qualities \cite{audio_digital_how_2014, dorincourt_reason_2010}.
Lastly, \textit{linear arithmetic synthesis} \cite{russ_l_1987} is a sound synthesis technique where recorded attacks of acoustic instruments are placed on the onset of synthesized tones, creating hybrids that have both acoustic and synthesized qualities.

The open and lightly touched string has been modelled and decomposed in the following papers.
Matti Karjalainen et al. in \cite{karjalainen_plucked-string_1998} decompose the model of the open string in modules for \textit{string excitation}, the \textit{string} itself, \textit{pickup mic} and \textit{bridge}.
In \cite{pakarinen_physical_2005}, Jyri Pakarinen adapts these modules to include a touching finger.
Guttler and Thelin in \cite{guettler_bowed-string_2012} take a different approach, providing an analysis of a touched string's impulse response, explaining step-by-step how its waveform is constructed.

\section{The lightly touched string}

This paper relies on \textit{digital waveguide} (DWG) synthesis \cite{smith_physical_1992}, a \textit{physical modelling} technique that employs D'Alambert's principle to analyse a standing wave (e.g. a vibrating string) into the sum of two travelling waves propagating in opposite directions.
All parts of the model are described in terms of traveling waves which are combined in the final step to produce the physical motion of the string.
Propagation losses are modelled with filters, and termination reflections are implemented with a direction inversion, a reflection loss filter and a sign inversion (often implemented as part of the loss filter).
Figure \ref{fig:pickup} depicts a DWG model of a simple plucked string with a pickup output.
Although the string touching agent may vary, it will be referred to as the \textit{finger}.
The terms \textit{bridge} and \textit{nut} will be used for the left and the right termination, respectively.

\begin{figure}[h]
	\centering
	\scalebox{0.75}{\begin{tikzpicture}[auto, node distance=3cm]
		% line 0
		\node[block, minimum width=60mm]	(delay0)	{$N/2$ samples delay};
		
		% line 1
		\node[block, below left = 0.8cm and 0cm of delay0, label={135:BRIDGE}]	(filter1a)	{Reflection \\ filter};
		\node[block, below right = 0.8cm and 0cm of delay0, label={45:NUT}]	(filter1b)	{Reflection \\ filter};
		\node[sum, left of=filter1b, xshift=-2.5cm]	(sum1) 	{};

		% line 2
		\node[block, below right = 0.8cm and 0cm of filter1a, minimum width=60mm]	(delay2)	{$N/2$ samples delay};

		\coordinate [above=1cm of sum1] (out0);
		\coordinate [right=1cm of sum1] (out1);
		\coordinate [below=1cm of sum1] (out2);

		% connections
		\path (delay0)		-|	(filter1b);
		\path (filter1b)	|-	(delay2);
		\path (delay2)		-|	(filter1a);
		\path (filter1a)	|-	(delay0);

		\begin{pgfonlayer}{bg}	% select the background layer
			\path (out0) -- (sum1);
			\path (out2) -- (sum1);
			\path (sum1) -- (out1) node [pos=2, text width=2.5cm, above] {Output};
		\end{pgfonlayer}
\end{tikzpicture}}
	\caption{
		A digital waveguide model of a string with a pickup output.
		Where $N$ is the duration of a roundtrip.
	}
	\label{fig:pickup}
\end{figure}

The touching point has been modelled in DWG synthesis as a three-port scattering-junction (SJ) \cite{scavone_digital_1997, valimaki_modeling_1993}.
This splits the model into three waveguides meeting at the SJ, one leading to the \textit{nut}, one leading to the \textit{bridge} and one leading to the \textit{finger}.
Since the \textit{finger} waveguide contains only a lossless reflection the SJ can be simplified as suggested in \cite{pakarinen_physical_2005}.
A string model with a simplified \textit{finger} SJ is depicted in fig. \ref{fig:waveguide_sj}.
The SJ coefficients $\rho_1 - \rho_4$ are defined in \cite{pakarinen_physical_2005} as

\begin{align} \label{eq:rhoCoeff}
	\rho_1 & = \frac{Z_1-Z_2-Z_3}{Z_1+Z_2+Z_3}, &
	\rho_2 & = \frac{2Z_2}{Z_1+Z_2+Z_3},\notag    \\[1em]
	\rho_3 & = \frac{2Z_1}{Z_1+Z_2+Z_3},        &
	\rho_3 & = \frac{Z_2-Z_1-Z_3}{Z_1+Z_2+Z_3},
\end{align}
where $Z_1$, $Z_2$ and $Z_3$ denote the impedance of the \textit{bridge}, \textit{nut} and \textit{finger} waveguides, respectively.
As the \textit{bridge} and \textit{nut} waveguides are part of the same string, their impedances should be equal, thus we have $Z_1=Z_2$.
Consequently, this equality leads to $\rho_1=\rho_4$ and $\rho_2=\rho_3$, leaving only coefficients $\rho_1$ and $\rho_2$.
Both can be expressed in relation to

\begin{equation} \label{eq:rho}
	\rho=\frac{Z_3}{2Z_1+Z_3}
\end{equation}
as
\begin{equation}`'
	\rho_1=-\rho \text{ and } \rho_2=1-\rho.
\end{equation}
The impedance $Z_3$ of the \textit{finger} waveguide varies depending on the pressure applied to the string.
The state of the SJ should span from disabled (no \textit{finger}) to a total reflection (the \textit{finger} acting as a termination), this implies that $0 \leq \rho \leq 1$.

\begin{figure}[h]
	\centering
	\scalebox{0.56}{\begin{tikzpicture}[auto, node distance=1cm]
	% line 0
	\node[block, minimum width=20mm]	(delay0a)	{$F/2$ delay};
	\node[branch, right = 1.5cm of delay0a]	(branch0)	{};
	\node[multiply right, label=above:$\rho_2$, right = 1cm of branch0, xshift=0.5cm]	(mul0)	{};
	\node[sum, right = 1cm of mul0]	(sum0)	{};
	\node[block, right = 1.5cm of sum0]	(delay0b)	{$(N-F)/2$ delay};

	% line 1
	\node[block, below left = 1cm and 0cm of delay0a]	(filter1a)	{Reflection \\ filter};
	\node[multiply down, label=left:$\rho_1$, below = 1cm of branch0, yshift=-0.5cm]	(mul1a)	{};
	\node[multiply up, label=right:$\rho_1$, below = 1cm of sum0]	(mul1b)	{};
	\node[block, below right = 1cm and 0cm of delay0b]	(filter1b)	{Reflection \\ filter};

	% line 2
	\node[block, below right = 1cm and 0cm of filter1a, minimum width=20mm]	(delay2a)	{$F/2$ delay};
	\node[sum, below = 1cm of mul1a]	(sum2)	{};
	\node[multiply left, label=below:$\rho_2$, right = 1cm of sum2]	(mul2)	{};
	\node[branch, below = 1cm of mul1b, yshift=-0.5cm]	(branch2)	{};
	\node[block, right = 1.5cm of branch2]	(delay2b)	{$(N-F)/2$ delay};

	\draw[dashed] (1.5,1) rectangle (7.5,-4.8);
	\node[above = 0.7cm of mul0] (sj) {Finger scattering-junction};

	\node[sum, left of=filter1a, xshift=2.5cm]	(sum1) 	{};
	\coordinate [above=1.5cm of sum1] (out0);
	\coordinate [below=1.5cm of sum1] (out2);
	\coordinate [right=1cm of sum1] (out1);
	\begin{pgfonlayer}{bg}	% select the background layer
		\path (out0) -- (sum1);
		\path (out2) -- (sum1);
		\path (sum1) -- (out1) node [pos=1.5, text width=2.5cm, above] {Output};
	\end{pgfonlayer}

	% Loop connections
	\path (delay0a)		--	(branch0);
	\path (branch0)		--	(mul0);
	\path (mul0)			--	(sum0);
	\path (sum0)			--	(delay0b);
	\path (delay0b)		-|	(filter1b);
	\path (filter1b)	|-	(delay2b);

	\path (delay2b)		--	(branch2);
	\path (branch2)		--	(mul2);
	\path (mul2)			--	(sum2);
	\path (sum2)			--	(delay2a);
	\path (delay2a)		-|	(filter1a);
	\path (filter1a)	|-	(delay0a);

	% Branch connections
	\path (branch0)		--	(mul1a);
	\path (mul1a)	--	(sum2);
	\path (branch2)		--	(mul1b);
	\path (mul1b)	--	(sum0);

	\draw[draw=red, opacity=0.4, line width=1mm] (-1.7,-3.5) rectangle ++(4.1,3.3);
	\draw[draw=green, opacity=0.4, line width=1mm] (6.5,-3.5) rectangle ++(4.9,3.3);
	\draw[draw=blue, opacity=0.4, line width=1mm] (-1.5,-3.3) rectangle ++(12.7,2.9);
\end{tikzpicture}}
	\caption{A digital waveguide model with a scattering junction.
		\textit{Bridge-finger} loop (red), \textit{finger-nut} loop (green) and \textit{bridge-nut} loop (blue)}
	\label{fig:waveguide_sj}
\end{figure}

To reduce the computational complexity of the proposed model, the scattering of waves from the nut to the finger is disregarded.
This means that no reflections or coefficient multiplications occur from \textit{nut} to \textit{bridge}.
This SJ will be referred to as the \textit{single-sided} SJ, a depiction can be seen in fig. \ref{fig:waveguide_simple_sj}.
Although this might seem like a crude simplification, comparing audio outputs reveals that the only discernible effect of the \textit{single-sided} SJ is a ``metallic'' or ``rough'' quality, which is an acceptable compromise for the purpose of this paper.
\begin{figure}[h]
	\centering
	\scalebox{0.6}{\begin{tikzpicture}[auto, node distance=1cm]
	% line 0
	\node[block, minimum width=30mm]	(delay0a)	{$F/2$ delay};
	\node[branch, right = 1.5cm of delay0a]	(branch0)	{};
	\node[multiply right, label=above:$\rho_2$, right = 1cm of branch0, xshift=0.5cm]	(mul0)	{};
	\node[block, right = 1.2cm of mul0]	(delay0b)	{$(N-F)/2$ delay};

	% line 1
	\node[block, below left = 1cm and 0cm of delay0a]	(filter1a)	{Reflection \\ filter};
	\node[multiply down, label=left:$\rho_1$, below = 1cm of branch0, yshift=-0.5cm]	(mul1a)	{};
	\node[block, below right = 1cm and 0cm of delay0b]	(filter1b)	{Reflection \\ filter};

	% line 2
	\node[block, below right = 1cm and 0cm of filter1a, minimum width=30mm]	(delay2a)	{$F/2$ delay};
	\node[sum, below = 1cm of mul1a]	(sum2)					{};
	\node[block, below left = 1cm and 0cm of filter1b]	(delay2b)		{$(N-F)/2$ delay};

	\draw[dashed] (2,1) rectangle (6,-4.8);
	\node[above = 0.7cm of mul0, xshift=-1cm] (sj) {Simplified scattering-junction};

	\node[sum, left of=filter1a, xshift=3cm]	(sum1) 	{};
	\coordinate [above=1.5cm of sum1] (out0);
	\coordinate [below=1.5cm of sum1] (out2);
	\coordinate [right=1cm of sum1] (out1);
	\begin{pgfonlayer}{bg}	% select the background layer
		\path (out0) -- (sum1);
		\path (out2) -- (sum1);
		\path (sum1) -- (out1) node [pos=1.5, text width=2.5cm, above] {Output};
	\end{pgfonlayer}


	% Loop connections
	\path (delay0a)		--	(branch0);
	\path (branch0)		--	(mul0);
	\path (mul0)			--	(delay0b);
	\path (delay0b)		-|	(filter1b);
	\path (filter1b)	|-	(delay2b);

	\path (delay2b)		--	(sum2);
	\path (sum2)			--	(delay2a);
	\path (delay2a)		-|	(filter1a);
	\path (filter1a)	|-	(delay0a);

	% Branch connections
	\path (branch0)		--	(mul1a);
	\path (mul1a)	--	(sum2);
\end{tikzpicture}}
	\caption{A digital waveguide with a simplified scattering junction.}
	\label{fig:waveguide_simple_sj}
\end{figure}

As previously mentioned, the SJ is the meeting point for three waveguides.
This results in the formation of one loop for each combination of waveguides, namely the \textit{bridge-finger} loop, the \textit{bridge-nut} loop, and the \textit{finger-nut} loop.
Using the \textit{single-sided} SJ eliminates the \textit{finger-nut} loop, leaving only the \textit{bridge-finger} and the \textit{bridge-nut} loop, also referred to as $F$- and $N$-roundtrips.

\section{explicit model}
% wavetable synthesis, phase accumulator, unipolar sawtooth wave
In \cite{pakarinen_physical_2005} a touched string DWG model is decomposed to three parts, namely, \textit{plucking position}, \textit{touched string}, and \textit{pickup output}.
The touched string part can be used as a starting point for obtaining a model for only the finger.
At first, the \textit{finger-nut} loop needs to be removed.
Figure \ref{fig:sdl_simple_sj} depicts an \textit{string loop} model that includes the touching finger using a \textit{single-sided} SJ, it consists of two coupled loops and their corresponding coefficients.
By detaching the finger ($\rho = 1$), it's easy to distinguish what parts are relevant to the finger (red) and to the string (blue).
This depiction, as most DWG models, shows a recursive process, where the effect of the string and the finger is applied to previous results.
In order to isolate the overall effect of a finger to a string, an explicit formula is proposed, where string looping, losses and finger coefficients are applied to the initial excitation signal.

\begin{figure*}[h]
	\centering
	\scalebox{0.8}{\begin{tikzpicture}[auto, node distance=1.5cm]

	% line 2
	\node[block]	(filter2)	{$N-F$ Filter};
	\node[block, right = 1cm of filter2]	(delay2)	{Delay $N-F$};
	\node[multiply left, right = 1cm of delay2, label={below:$1-\rho$}]		(mul2)		{};

	% line 1
	\node[sum, left of=filter2, above of=filter2]			(sum1)		{};
	\node[branch, right of=mul2, above of=mul2]	(branch1)	{};
	\node[multiply left, label=above:{$\rho$}]					(mul1)		at ($(sum1)!1/2!(branch1)$)	{};
	\node[block, right = 1cm of branch1]					(filter1)	{$F$ Filter};
	\node[block, right = 1cm of filter1]					(delay1)	{Delay $F$};

	% line 0
	\node[sum, left of=sum1, above of=sum1]					(sum0)		{};
	\node[branch, right of=delay1, above of=delay1]	(branch0)	{};

	\coordinate[left=1cm of sum0]			(input);
	\coordinate[right=1cm of branch0]	(output);

	\path (delay1)		--	(filter1);
	\path (filter1)		--	(branch1);
	\path (mul2)			--	(delay2);
	\path (delay2)		--	(filter2);
	\path (filter2)		-|	(sum1);

	\path (branch1)		|-	(mul2);
	\path (branch1)		--	(mul1);
	\path (mul1)	--	(sum1);

	\path (sum0)			--	(branch0);
	\path (branch0)		|-	(delay1);
	\path (sum1)			-|	(sum0);

	\path[dashed] (branch0)		--	(output);
	\path[dashed] (input)			--	(sum0);
\end{tikzpicture}}
	\caption{String loop with a scattering junction. $N$-roundtrip (blue), $F$-roundtrip (red)}
	\label{fig:sdl_simple_sj}
\end{figure*}

% similar to the approach taken in \cite{guettler_bowed-string_2012}.

For an initial excitation signal $y_{\textrm{in}}(t)$, the traveling wave of a touched string is given by

\begin{equation} \label{eq:explicit}
	y_{\textrm{out}}(t) = \sum_{k=0}^{\lfloor \frac{t}{T_F} \rfloor}\left(y_{\textrm{in}}(t - kT_F - mT_N)\binom{m+k}{k} (1 - \rho)^{k} \rho^mR_N^mR_F^k\right),
\end{equation}
\begin{equation*}
	\textrm{for } T_F<T_N<t \textrm{ and } 0\leq\rho\leq1\notag,
\end{equation*}
Here, the variables and terms used are defined as follows:
\begin{itemize}
	\setlength\itemsep{0.1em}
	\item $t$ denotes the current time,
	\item $T_N$ and $T_F$ denote the duration of the $N$ and $F$ roundtrips, respectively,
	\item $\rho$ is the finger pressure coefficient defined in (\ref{eq:rho}),
	\item $R_N$ and $R_F$ denote the reflection and propagation losses of $N$ and $F$ roundtrips, respectively, and
	\item $m$ is defined as
\end{itemize}
\begin{equation}\label{eq:m}
	m = \mmm.	
\end{equation}
The initial values are given by $y_{\textrm{in}}(t)$ for $0 \leq t < T_N$.
Each addend of the sum calculates the signal produced after $k$ $F$-roundtrips and $m$ $N$-roundtrips in a given time $t$.
The argument of the summation in (\ref{eq:explicit}) can be split into four parts,
\begin{flalign}
	\label{eq:shifts} & \bullet\qquad\textrm{shifts }       &  & y_{\textrm{in}}(t - kT_F - mT_N),       \\[1em]
	\label{eq:perm}   & \bullet\qquad\textrm{permutations } &  & \binom{m+k}{k},        &  & \\[1em]
	\label{eq:coef}   & \bullet\qquad\textrm{coefficients } &  & (1 - \rho)^{k} \rho^m, &  & \\[1em]
	\label{eq:loss}   & \bullet\qquad\textrm{and losses }   &  & R_N^mR_F^k.            &  &
\end{flalign}
Firstly, (\ref{eq:shifts}) loops the excitation signal by shifting it forward in time by an amount proportional to the number of $N$- and $F$-roundtrips performed.
Variables $m$ and $k$ work as counters for the number of complete $N$ and $F$-roundtrips respectively.
A depiction of the shifts can be seen in Figure \ref{eq:shifts}.
The first (\ref{fig:shifts}.1) plot starts with the initial excitation signal (in bold), followed by shifts of $N$-roundtrips, i.e. multiples of $T_N$.
Since $T_F<T_F$, $F$-roundtrip shifts will fall on top of $N$-roundtrip shifts in the current depiction (dotted lines).
In order to avoid a confusing illustration, shifts starting with a single $F$-roundtrip are moved to the above plot (\ref{fig:shifts}.2), shifts starting with two $F$-roundtrips are moved to plot (\ref{fig:shifts}.3) and so on.
Figure (\ref{fig:triangle}) is a concise form of figure (\ref{eq:shifts}), where the initial excitation signal and each of it's shifts are enclosed in a box, forming a "brick wall triangle".
This depiction is going to be used as a basis for showing following figures.

\begin{figure}[h]
	\centering
	\scalebox{0.75}{\begin{tikzpicture}[xscale=2, samples=100]
	\tikzmath {
		% params
		\tf = 0.62;
		\steps = 5;
		\lines = 4;
		% spacing
		\lineSpacing = 3.5;
		\bracesOffesY = -1.1;
		% computed
		\lineMax = \lines-1;
	}
	% initial plot
	\draw[-, color=black, domain=0:1, line width=1mm] plot (\x,{sin(\x*2*pi r)});
	\foreach \lineIndex in {0, ..., \lineMax} {
			\tikzmath {
				int \lineNum, \fullStepsPerLine;
				\linePosition = \lineIndex * \lineSpacing;
				\lineNum = \lineIndex + 1;
				\fullStepsPerLine = floor(\steps - \lineIndex*\tf);
			}
			\ifnum \fullStepsPerLine>0
				% axes
				\draw[-] (-0.2,\linePosition) node[left] {(\lineNum)} -- (\steps,\linePosition) node[right] {$t$};
				\draw[-] (0,-1.2+\linePosition) -- (0,1.2+\linePosition) node[above] {$A$};
				% plot
				\draw[-, color=black, domain=0:\steps-\lineIndex*\tf] plot (\lineIndex*\tf+\x,{sin(\x*2*pi r)+\linePosition});
				% dotted plot on the first line
				\draw[-, color=gray, dotted, thin, domain=0:\steps-\lineIndex*\tf] plot (\lineIndex*\tf+\x,{sin(\x*2*pi r)});
				% exclude the first plot
				\ifnum \lineIndex>0
					% T_F brackets
					\foreach \tfPerLine in {1, ..., \lineIndex} {
							\draw[|-|, thin, color=red] (\tfPerLine*\tf-\tf, \bracesOffesY+\linePosition) -- (\tfPerLine*\tf, \bracesOffesY+\linePosition) node[midway, below] () {$T_F$};
						}
					% arrows
					\draw[->, thin, color=gray] (1/4+\lineIndex*\tf, 1+0.1) -- (1/4+\lineIndex*\tf, \lineIndex*\lineSpacing-0.5);
					% T_F brackets and dividers on the fist line
					\draw[|-|, thin, color=red] (\lineIndex*\tf-\tf, \bracesOffesY*2) -- (\lineIndex*\tf, \bracesOffesY*2) node[midway, above] () {$T_F$};
					\draw[-, dashed, thin, color=red] (\lineIndex*\tf, \bracesOffesY*2) -- (\lineIndex*\tf, 1);
				\fi
				% tn grid
				\tikzmath{
					int \fullStepsPerLineMax;
					\fullStepsPerLineMax = \fullStepsPerLine-1;
				}
				\foreach \stepIndex in {0, ..., \fullStepsPerLineMax} {
						% brackets
						\draw[|-|, thin, color=blue] (\lineIndex*\tf+\stepIndex, \bracesOffesY+\linePosition) -- (\lineIndex*\tf+\stepIndex+1, \bracesOffesY+\linePosition) node[midway, below] () {$T_N$};
						% dividers (left and right)
						\draw[-, very thin, blue] (\lineIndex*\tf+\stepIndex+1, 1+\linePosition) -- (\lineIndex*\tf+\stepIndex+1, \bracesOffesY+\linePosition);
						\draw[-, very thin, gray] (\lineIndex*\tf+\stepIndex, 1+\linePosition) -- (\lineIndex*\tf+\stepIndex, \bracesOffesY+\linePosition);
					}
			\fi
		}

\end{tikzpicture}


}
	\caption{
		Shifts of the initial excitation signal (bold), split in separate plots according to the number of $F$-roundtrips.
	}
	\label{fig:shifts}
\end{figure}

\begin{figure}[h]
	\centering
	\tri{plot}
	\caption{
		The brick wall triangle, a compact version of figure \ref{fig:shifts} where each shift is enclosed in a box.
	}
	\label{fig:triangle}
\end{figure}

Combinations of roundtrips are handled by the shifts, however, permutations within each combination are not currently considered.
Let's consider a case where $k = 2$ and $m = 1$.
In this case, three permutations of roundtrips are performed concurrently, namely $FFN$, $FNF$, and $NFF$.
It is important to note that since $k$ and $m$ remain the same, each permutation yields an identical signal.
Therefore, obtaining the sum of all three permutations is simply a matter of tripling the amplitude.
Generalizing, accounting for permutations is achieved by multiplying the shifted signal by the number of allowed permutations.
This number can be obtained using the binomial coefficient as shown in (\ref{eq:perm}).
On the brick wall triangle, permutation coefficients can be obtained by counting the number of paths leading from the initial excitation to the shift in question (figure \ref{fig:permutations}).
Also, if the permutation coefficient of each shift is written in the corresponding box, the triangle becomes a slightly stretched version of Pascal's triangle turned on it's left side.

\begin{figure}[h]
	\centering
	\tri{permutations}
	\caption{
		The brick wall triangle with each box containing the number of permutations for the 
	}
	\label{fig:permutations}
\end{figure}

Each roundtrip involves a SJ coefficient as defined in (\ref{eq:rho}).
The cumulative effect of the coefficients, for a given $k$ and $m$ can be computed as demonstrated in (\ref{eq:coef}).

Lastly, (\ref{eq:loss}) represents the roundtrip losses for any given $k$ and $m$.
The loss implementation might vary depending on the modelling approach, in this case we choose a plain loss coefficient which can be potentially replaced by a more elaborate loss model.

\section{Multiphonics abstraction}

This section presents various modifications of (\ref{eq:explicit}) capable of receiving the traveling wave of an untouched string, or an arbitrary periodic signal as input.
Firstly, we obtain the formula of the untouched string $(y_{\textrm{nof}})$ using (\ref{eq:explicit}) and setting $\rho=1$.
This causes every addend in the summation (\ref{eq:explicit}) where $k > 0$, to evaluate to zero.
Therefore, the original sum reduces to just the fist term where $k = 0$.

\begin{equation}\label{eq:rho0}
	y_{\textrm{nof}}(t) = y_{\textrm{in}}(t-\lfloor\frac{t}{T_N}\rfloor T_N)R_N^{\lfloor\frac{t}{T_N}\rfloor}
\end{equation}

Note that the argument of $y_{in}$ has the form of a sawtooth wave with a period of $T_N$.
Therefore, $y_{\textrm{nof}}$ has a period of $T_N$ and each cycle undergoes the losses of a $T_N$ roundtrip ($R_N$).
Using the definition of $y_{\textrm{nof}}$ we can rewrite (\ref{eq:explicit}) as follows

\begin{equation} \label{eq:y0rf}
	y_{\textrm{out}}(t) = \sum_{k=0}^{\lfloor \frac{t}{T_F} \rfloor}\left(y_{\textrm{nof}}(t - kT_F)\binom{m+k}{k} (1 - \rho)^{k} \rho^mR_F^k\right)
\end{equation}

See Appendix \ref{appendix:math1} for a derivation.

This formula takes a traveling wave of an open string as input ($y_{\textrm{nof}}$) and converts it to the traveling wave of an identical string with an additional touching finger.
By treating an arbitrary signal as the traveling wave of an open string, the formula can be used to add the effect of the touching finger to synths or sounds of other instruments.
The only limitation is that $y_{\textrm{nof}}$ needs to be periodic with a period of $T_N$.

One term needing clarification is $R_F$.
The losses of the finger loop ($R_F$) should be related to the losses of the open string ($R_N$) since both loops share the nut reflection and a part of string propagation.
A suitable value for $R_F$ should thus be derived from $R_N$.
This can be tricky if $y_{\textrm{nof}}$ comes from an external source (e.g. a recorded string) where $R_N$ is unknown.
An approximation would be achievable comparing consecutive cycles of $y_{\textrm{nof}}$, however, the need for finding $R_N$ can be avoided altogether if $R_F = R_N$.
This is a reasonable simplification since the two losses should be relatively close, especially when $T_F \approx T_N$, when the roundtrips are mostly the same.
If the two losses are equal, the $n$-th cycle of $y_{\textrm{nof}}$ i.e. the cycle that has undergone $R_N^n$ losses, will be identical to cycles that have undergone $R_F^k$ and $R_N^m$ losses, where $k+m=n$.
Equal loss cycles occur at different points in time, this means that achieving the right losses is a matter of shifting $y_{\textrm{nof}}$.
Given that $R_F = R_N$, formula (\ref{eq:y0rf}) becomes

\begin{equation} \label{eq:y0}
	y_{\textrm{out}}(t) = \sum_{k=0}^{\lfloor \frac{t}{T_F} \rfloor}\left(y_{\textrm{nof}}(t - kT_F + kT_N)\binom{m+k}{k} (1 - \rho)^{k} \rho^m\right)
\end{equation}

giving us the appropriate shifts (see appendix \ref{appendix:math2}).

\begin{figure}[h]
	\centering
	\tri{losses}
	\caption{
		losses
	}
	\label{fig:losses}
\end{figure}

\section{Discussion}
This section points the limitations of the model and proposes future development steps to overcome them.
The model is not computationally sustainable for long durations due to the increasing number of concurrent shifts.
An input signal with a frequency of $\SI{440}{\hertz}$ ($T_N\approx\SI{2.27}{\milli\second}$) and a touching finger at the middle of the string ($T_F\approx\SI{1.14}{\milli\second}$) will reach $440$ concurrent shifts in half a second.
This number will increase linearly by $880$ shifts per second.
Since the losses and coefficients lower the amplitude of the signal on each cycle, resources could be freed after the amplitude decreases below a threshold, and repurposed for new shifts.

When the finger fully presses a physical string it shortens its active length, producing a higher pitch.
The model in this case should act as a pitch shifting algorithm, which it does, but in a very basic way, producing poor results.
The model could be improved by adopting elements of pitch shifting algorithms.

The shifts of the input signal are positive, this means that future signal is shifted to the current time.
This requires having future values of the input signal available beforehand, which can be done with a pre-recorded input signal, but makes real-time implementations impossible.
A number of approaches can be taken for a real-time version, such as using the most recent past signal instead of a future signal, or projecting the signal losses into the future and constructing a future signal approximation.

% REFERENCES FORMAT
% References must be the same font size as other body text.
\bibliographystyle{SIGCHI-Reference-Format}
\bibliography{bibliography.bib}

\clearpage
\appendix

\section{Appendix A}\label{appendix:math1}
The function of the untouched string

$y_{\textrm{nof}}(t) = y_{\textrm{in}}(t-\lfloor\frac{t}{T_N}\rfloor T_N)R_N^{\lfloor\frac{t}{T_N}\rfloor}$
(\ref{eq:rho0})

is part of the explicit formula of the touched string (\ref{eq:explicit}),
this relationship can 

forward in time by $-kT_F$

$y_{\textrm{nof}}(t-kT_F) = y_{\textrm{in}}(t-kT_F-\mmm T_N)R_N^{\mmm}$

and using $m=\mmm$ to obtain

$y_{\textrm{nof}}(t-kT_F) = y_{\textrm{in}}(t-kT_F-m T_N)R_N^{m}$.

In this form it is easy to recognize $y_{\textrm{nof}}$ inside

$y_{\textrm{out}}(t) = \sum_{k=0}^{\lfloor \frac{t}{T_F} \rfloor}\left(y_{\textrm{in}}(t - kT_F - mT_N)\binom{m+k}{k} (1 - \rho)^{k} \rho^mR_N^mR_F^k\right)$
(\ref{eq:explicit})

and substituted with $y_{\textrm{nof}}(t-kT_F)$, resulting in

$y_{\textrm{out}}(t) = \sum_{k=0}^{\lfloor \frac{t}{T_F} \rfloor}\left(y_{\textrm{nof}}(t - kT_F)\binom{m+k}{k} (1 - \rho)^{k} \rho^mR_F^k\right)$.


\section{Appendix B}\label{appendix:math2}
Starting from the expression of the untouched string

$y_{\textrm{nof}}(t) = y_{\textrm{in}}((\frac{t}{T_N}-\lfloor\frac{t}{T_N}\rfloor)T_N)R_N^{\lfloor\frac{t}{T_N}\rfloor}$,

we give a separate definition to the lossless part of the untouched string's function, as follows

$y_{\textrm{nol}}(t) = y_{\textrm{in}}((\frac{t}{T_N}-\lfloor\frac{t}{T_N}\rfloor)T_N)$,

and rewriting $y_{\textrm{nof}}$ accordingly

$y_{\textrm{nof}}(t) = y_{\textrm{nol}}(t)R_N^{\lfloor\frac{t}{T_N}\rfloor}$.

Assuming that $R=R_F=R_N$

$y_{\textrm{out}}(t) = \sum_{k=0}^{\lfloor \frac{t}{T_F} \rfloor}\left(y_{\textrm{nol}}(t - kT_F)\binom{m+k}{k} (1 - \rho)^{k} \rho^mR_N^{m}R_F^k\right)=$


$\sum_{k=0}^{\lfloor \frac{t}{T_F} \rfloor}\left(y_{\textrm{nol}}(t - kT_F)\binom{m+k}{k} (1 - \rho)^{k} \rho^mR^{m+k}\right)=$

$\sum_{k=0}^{\lfloor \frac{t}{T_F} \rfloor}\left(y_{\textrm{nol}}(t - kT_F)\binom{m+k}{k} (1 - \rho)^{k} \rho^mR^{\mmm+k}\right)=$

$\sum_{k=0}^{\lfloor \frac{t}{T_F} \rfloor}\left(y_{\textrm{nol}}(t - kT_F)\binom{m+k}{k} (1 - \rho)^{k} \rho^mR^{\left\lfloor \frac{t - kT_F + kT_N}{T_N} \right\rfloor}\right)=$

$\sum_{k=0}^{\lfloor \frac{t}{T_F} \rfloor}\left(y_{\textrm{nol}}(t - kT_F + kT_N)\binom{m+k}{k} (1 - \rho)^{k} \rho^mR^{\left\lfloor \frac{t - kT_F + kT_N}{T_N} \right\rfloor}\right)=$

$\sum_{k=0}^{\lfloor \frac{t}{T_F} \rfloor}\left(y_{\textrm{nof}}(t - kT_F + kT_N)\binom{m+k}{k} (1 - \rho)^{k} \rho^m\right)$



\end{document}

%%% Local Variables:
%%% mode: latex
%%% TeX-master: t
%%% End: